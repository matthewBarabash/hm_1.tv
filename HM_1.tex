\documentclass[a4paper,12pt]{article} % добавить leqno в [] для нумерации слева

%%% Работа с русским языком
\usepackage{cmap}					% поиск в PDF
\usepackage{mathtext} 				% русские буквы в фомулах
\usepackage[T2A]{fontenc}			% кодировка
\usepackage[utf8]{inputenc}			% кодировка исходного текста
\usepackage[english,russian]{babel}	% локализация и переносы
\usepackage[left=1cm,right=2cm,top=2cm,bottom=2cm,footskip=1cm,includefoot]
{geometry}
%%% Дополнительная работа с математикой
\usepackage{amsmath,amsfonts,amssymb,amsthm,mathtools} % AMS
\usepackage{icomma} % "Умная" запятая: $0,2$ --- число, $0, 2$ --- перечисление

%% Номера формул
\mathtoolsset{showonlyrefs=true} % Показывать номера только у тех формул, на которые есть \eqref{} в тексте.

%% Перенос знаков в формулах (по Львовскому)
\newcommand*{\hm}[1]{#1\nobreak\discretionary{}
	{\hbox{$\mathsurround=0pt #1$}}{}}

%%% Работа с картинками
\usepackage{graphicx}  % Для вставки рисунков
\graphicspath{{images/}{images2/}}  % папки с картинками
\setlength\fboxsep{3pt} % Отступ рамки \fbox{} от рисунка
\setlength\fboxrule{1pt} % Толщина линий рамки \fbox{}
\usepackage{wrapfig} % Обтекание рисунков и таблиц текстом

%%% Работа с таблицами
\usepackage{array,tabularx,tabulary,booktabs} % Дополнительная работа с таблицами

\usepackage{longtable}  % Длинные таблицы
\usepackage{multirow} % Слияние строк в таблице
\newcommand{\RomanNumeralCaps}[1]
{\MakeUppercase{\romannumeral #1}}
\newcommand{\tb}[1]{\textbf{#1}}
\usepackage{pgfplots, pgfplotstable}
\pgfplotsset{compat=1.9}
\newcommand{\ol}[1]{\ensuremath{\overline{#1}}}
\usepackage{circuitikz}

\begin{document} % конец преамбулы, начало документа

\hfill
\begin{minipage}{4cm}
	Просвирин Кирилл\\
	712 группа\\
\end{minipage}

\begin{center}
	\Large\textbf{Теория вероятностей\\} 
	\large Задание 1. Неделя 1
\end{center}

\section{Комбинация событий}

\paragraph{Т.1.} Пусть $ A, B, C $~--- три события. 
Найти выражения для событий:
\begin{enumerate}
	\item произошло только $ A $;
	\item произошли $ A $ и $ B $, а $ C $ не произошло;
	\item все три события произошли;
	\item произошло хотя бы одно из них;
	\item произошло только одно из них;
	\item ни одно из них не произошло;
	\item произошло не более двух;
\end{enumerate}

\textbf{Решение.}

\begin{enumerate}
	\item $ A\overline{B}\overline{C} $;
	\item $ AB\overline{C} $;
	\item $ ABC $;
	\item $ A~\cup~B~\cup~C $
	\item $ A~\overline{B}~\overline{C}~
	\cup~\overline{A}~B~\overline{C}~
	\cup~\overline{A}~\overline{B}~C $;
	\item $ \overline{A}~\cap~\overline{B}~\cap~\overline{C} $;
	\item $ \overline{ABC} $
\end{enumerate}

\paragraph{Т.2.} Пусть $ A, B $~--- два события. 
Найти все события $ X $ такие, что
\[
	\overline{(X~\cup~A)}~\cup~\overline{(X~\cup~\overline{A})}=B.
\]
$ \blacktriangleright $
\begin{align}&
	B=\overline{(X~\cup~A)}~\cup~\overline{(X~\cup~\overline{A})}=
	\ol{X}~\cap~\ol{A}~\cup~\ol{X}~\cap~A=
	\ol{X}~(\ol{A}~\cup~A)=\ol{X}~\Omega=\ol{X}~~~\Longrightarrow\\
	&X\cdot B=\varnothing~~~\Longleftrightarrow~~~X\subset\ol{B}	
\end{align}
\tb{Ответ: }$ X\subset\ol{B} $\hfill$ \blacktriangleleft $

\paragraph{Т.3.} Справедливы ли следующие равенства:

$\begin{array}{lc}
	1.~~\ol{A~\cap~B}=\ol{A}~\cup~\ol{B}; & 
	3.~~\ol{\ol{A}~\cap~\ol{B}}=A~\cup~B;\\
	2.~~\ol{A~\cup~B}=\ol{A}~\cap~\ol{B}; &
	4.~~\ol{\ol{A}~\cup~\ol{B}}=A~\cap~B;\\
	5.~~(A~\cup~B)~\cap~(A\cap\ol{B})~\cup~(\ol{A}~\cap~\ol{B}) &\\
	6.~~(A~\cup~B)\cap\ol{B}=A\cap\ol{B}=A~\cap~
\end{array}$


\end{document}




























