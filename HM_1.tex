\documentclass[a4paper,12pt]{article} % добавить leqno в [] для нумерации слева

%%% Работа с русским языком
\usepackage{cmap}					% поиск в PDF
\usepackage{mathtext} 				% русские буквы в фомулах
\usepackage[T2A]{fontenc}			% кодировка
\usepackage[utf8]{inputenc}			% кодировка исходного текста
\usepackage[english,russian]{babel}	% локализация и переносы
\usepackage[left=1cm,right=2cm,top=2cm,bottom=2cm,footskip=1cm,includefoot]
{geometry}
%%% Дополнительная работа с математикой
\usepackage{amsmath,amsfonts,amssymb,amsthm,mathtools} % AMS
\usepackage{icomma} % "Умная" запятая: $0,2$ --- число, $0, 2$ --- перечисление

%% Номера формул
\mathtoolsset{showonlyrefs=true} % Показывать номера только у тех формул, на которые есть \eqref{} в тексте.

%% Перенос знаков в формулах (по Львовскому)
\newcommand*{\hm}[1]{#1\nobreak\discretionary{}
	{\hbox{$\mathsurround=0pt #1$}}{}}

%%% Работа с картинками
\usepackage{graphicx}  % Для вставки рисунков
\graphicspath{{images/}{images2/}}  % папки с картинками
\setlength\fboxsep{3pt} % Отступ рамки \fbox{} от рисунка
\setlength\fboxrule{1pt} % Толщина линий рамки \fbox{}
\usepackage{wrapfig} % Обтекание рисунков и таблиц текстом

%%% Работа с таблицами
\usepackage{array,tabularx,tabulary,booktabs} % Дополнительная работа с таблицами

\usepackage{longtable}  % Длинные таблицы
\usepackage{multirow} % Слияние строк в таблице
\newcommand{\RomanNumeralCaps}[1]
{\MakeUppercase{\romannumeral #1}}
\newcommand{\tb}[1]{\textbf{#1}}
\newcommand{\abs}[1]{\left|#1\right|}
\newcommand{\ol}[1]{\ensuremath{\overline{#1}}}

\usepackage{pgfplots, pgfplotstable}
\pgfplotsset{compat=1.9}
\usepackage{circuitikz}
\usepackage{ulem}
\usepackage{cancel}

\begin{document} % конец преамбулы, начало документа

\hfill
\begin{minipage}{4cm}
	Просвирин Кирилл\\
	712 группа\\
\end{minipage}

\begin{center}
	\Large\textbf{Теория вероятностей\\} 
	\large Задание 1. Неделя 1
\end{center}

\section{Комбинация событий}

\paragraph{Т.1.} Пусть $ A, B, C $~--- три события. 
Найти выражения для событий:
\begin{enumerate}
	\item произошло только $ A $;
	\item произошли $ A $ и $ B $, а $ C $ не произошло;
	\item все три события произошли;
	\item произошло хотя бы одно из них;
	\item произошло только одно из них;
	\item ни одно из них не произошло;
	\item произошло не более двух;
\end{enumerate}

\textbf{Решение.}

\begin{enumerate}
	\item $ A\overline{B}\overline{C} $;
	\item $ AB\overline{C} $;
	\item $ ABC $;
	\item $ A~\cup~B~\cup~C $
	\item $ A~\overline{B}~\overline{C}~
	\cup~\overline{A}~B~\overline{C}~
	\cup~\overline{A}~\overline{B}~C $;
	\item $ \overline{A}~\cap~\overline{B}~\cap~\overline{C} $;
	\item $ \overline{ABC} $
\end{enumerate}

\paragraph{Т.2.} Пусть $ A, B $~--- два события. 
Найти все события $ X $ такие, что
\[
	\overline{(X~\cup~A)}~\cup~\overline{(X~\cup~\overline{A})}=B.
\]
$ \blacktriangleright $
\begin{align}&
	B=\overline{(X~\cup~A)}~\cup~\overline{(X~\cup~\overline{A})}=
	\ol{X}~\ol{A}~\cup~\ol{X}A=
	\ol{X}~(\ol{A}~\cup~A)=\ol{X}~\Omega=\ol{X}~~~\Longleftrightarrow\\
	&\Longleftrightarrow~~~
	\begin{cases}
		X\cdot B=\varnothing\\
		\ol{X}\cdot \ol{B}=\varnothing
	\end{cases}~~~\Longleftrightarrow~~~
	\begin{cases}
		X\subset\ol{B}\\
		\ol{X}\subset B
	\end{cases}~~~\Longleftrightarrow~~~
	\begin{cases}
		X\subset\ol{B}\\
		\ol{B}\subset X
	\end{cases}	
\end{align}
\tb{Ответ: }$ X\subset\ol{B};~~\ol{B}\subset X $
\hfill$ \blacktriangleleft $

\paragraph{Т.3.} Справедливы ли следующие равенства:

$\begin{array}{lc}
	1.~~\ol{A~\cap~B}=\ol{A}~\cup~\ol{B}; & 
	2.~~\ol{\ol{A}~\cap~\ol{B}}=A~\cup~B;\\
	3.~~\ol{A~\cup~B}=\ol{A}~\cap~\ol{B}; &
	4.~~\ol{\ol{A}~\cup~\ol{B}}=A~\cap~B;\\
	5.~~(A~\cup~B)~\cap~\ol{A~\cap~B}=
	(A\cap\ol{B})~\cup~(\ol{A}~\cap~B); &\\
	6.~~(A~\cup~B)\cap\ol{B}=A\cap\ol{B}=A~\cap~\ol{A~\cap~B}; &\\
	7.~~\ol{A}~\cap~\ol{B}=A~\cup~B
\end{array}$\\

\noindent$ \blacktriangleright $~Первые 4 равенства очевидно справдливы
по закону де Моргана. 
\begin{enumerate}
	\setcounter{enumi}{4}
	\item 
	$
		(A~\cup~B)~\cap~\ol{A~\cap~B}=
		(A~\cup~B)~\cap~(\ol{A}~\cup~\ol{B})=
		\cancel{A\ol{A}}~\cup~A\ol{B}~\cup~B\ol{A}~\cup~\cancel{B\ol{B}}=
		A\ol{B}~\cup~B\ol{A}.
	$
	
	\tb{Ответ: } Справедливо.
	\item $ (A~\cup~B)\cap\ol{B}=A\ol{B}~\cup~\cancel{B\ol{B}}=A\ol{B} $.
	
	С друго стороны: 
	$ A~\cap~\ol{A~\cap~B}=A~\cap~(\ol{A}~\cup~\ol{B})=
	\cancel{A\ol{A}}~\cup~A\ol{B}=A\ol{B} $.
	
	\tb{Ответ: } Справедливо.
	\item $ \ol{A}~\cap~\ol{B}=\ol{A~\cup~B}\ne A~\cup~B $.
	
	\tb{Ответ: } Несправедливо.\hfill$ \blacktriangleleft $
\end{enumerate}

\paragraph{Т.4.} Найти простые выражения для событий
\[
	1)~(A~\cup~B)~\cap~(A~\cup~\ol{B});~~
	2)~(A~\cup~B)~\cap~(\ol{A}~\cup~\ol{B})~\cap~(A~\cup~\ol{B});~~
	3)~(A~\cup~B)~\cap~(B~\cup~C)
\]
$ \blacktriangleright $
\begin{enumerate}
	\item $ (A~\cup~B)(A~\cup~\ol{B})=
	A~\cup~A\ol{B}~\cup~AB~\cup~\cancel{B\ol{B}}=
	A~\cup~A(B~\cup~\ol{B})=\boxed{A} $
	\item $ (A~\cup~B)(\ol{A}~\cup~\ol{B})~\cap~(A~\cup~\ol{B})=
	(A\ol{B}~\cup~\ol{A}B)(A~\cup~\ol{B})=\boxed{A\ol{B}}$
	\item $ (A~\cup~B)(B~\cup~C)=AB~\cup~AC~\cup~B~\cup~BC=
	\boxed{A(B~\cup~C)~\cup~B} $\hfill$ \blacktriangleleft $
\end{enumerate}

\paragraph{Т.5.} Что вероятнее, получить хотя бы одну единицу при
бросании четырех игральных костей или хотя бы одну пару единиц при 
24 бросаниях двух костей?\\

\noindent$ \blacktriangleright $ Для начала подсчитаем вероятность получить
хотя бы одну единицу
\begin{align}
	\begin{cases}
		\abs{\Omega}=6^4\\
		\abs{\ol{A}_1}=5^4
	\end{cases}~~~\Longrightarrow~~~P(A_1)=
	1-\dfrac{5^4}{6^4}>0,5
\end{align}

Вероятность получить одну пару единиц
\begin{align}
	\begin{cases}
		\abs{\Omega}=36^{24}\\
		\abs{\ol{A}_2}=35^{24}
		\end{cases}~~~\Longrightarrow~~~P(A_1)=
	1-\dfrac{35^{24}}{36^{24}}<0,5
\end{align}
\tb{Ответ: }Вероятнее получить хотя бы одну единицу при
бросании четырех игральных костей.\hfill$ \blacktriangleleft $

\paragraph{Т.6.}



\end{document}










